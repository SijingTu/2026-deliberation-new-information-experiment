\section{Sequential deliberation under social pressure}
We will use the following notation. 
We consider a set of $n$ agents, denoted by $N$, indexed by $u \in [n]$. 
Consider a set of $m$ alternatives, denoted by $M$, indexed by $i \in [m]$. 
Now, we consider each agent has a bliss point in the set of alternatives, denoted by $a^*_u \in M$. 
The bliss point is not involved in the deliberation process, instead, we assume that the agent may uses another stance. 

For instance, we first consider an initial setup. 
At the first time that the agent $u$ participates in the deliberation, and it encounters the alternative $o$, the agent $u$ first updates its stance to $a_u = \phi(\{a_u^*, o\})$. 
The function $\phi$ is a function that we may specify later, for instance, when the graph is a line, the function $\phi$ can be the one that takes the midpoint of the bliss point and the alternative. 
At the initial step, we assume that the agent does not participate in the bargining, but only selects its stance. 

Suppose at the $t$-th round of the deliberation, the agent $u$ uses the stance $a_u^t \in M$. 
Let $o^t$ be the outcome of the $t$-th round of the deliberation.  
The deliberation process is as follows. 

\begin{enumerate}
    \item Iteration over $t = 1, 2, \ldots, T$:
    \begin{enumerate}
        \item Uniformly randomly selects a pair of agents $i$ and $j$ from $N$. Let $a_i^t$ and $a_j^t$ be the stances that the agents $i$ and $j$ use at the $t$-th round for deliberation. 
        \item The outcome of the previous round of deliberation is $o^{t-1}$. The outcome of the current round of deliberation is $o^t$ is the result of the Nash bargaining between the agents $i$ and $j$ using the stances $a_i^t$ and $a_j^t$ for deliberation, i.e., $o^t = Median(\{a_i^t, a_j^t, o^{t-1}\})$ in the median graph setting.
        \item Agent $i$ updates its stance to $a_i^t = Median(\{a_i^*, a_i^{t-1}, o^{t}\})$; and agent $j$ updates its stance to $a_j^t = Median(\{a_j^*, a_j^{t-1}, o^{t}\})$. For all the other agents $u \in N \setminus \{i, j\}$, their stances are not updated, i.e., $a_u^t = a_u^{t-1}$.
    \end{enumerate}
    \item The final outcome is $o^T$. 
\end{enumerate}

In the above setting, we assume that the stance that the agent ultilizes in the bargaining is the outcome of both the agent's true stance (bliss point), the current alternative in the bargaining, and the stance that the agent uses in the previous bargaining. 

\subsection{Observation}
The stances that the agents choose always shrink towards the bliss points. 
However, it is not entirely guaranteed that the stances will eventually converge to the bliss points. (In the experiment, it does not seem to be the case.)